% document
\documentclass[10pt,graphics,aspectratio=169,table]{beamer}
\usepackage{listings}
\usepackage{csquotes}
\usepackage{hyperref}
\usepackage{xcolor}
% theme
\usetheme{metropolis}
% packages
\title{Lesson 2}
\author{Christian Schwarz, Jakob Krebs}
\date{04.11.2019}
\begin{document}
\maketitle

\begin{frame}{Contents}
	\tableofcontents
\end{frame}

\lstset{
    showstringspaces=false,
    basicstyle=\ttfamily,
    keywordstyle=\color{blue},
    commentstyle=\color[green]{0.6},
    stringstyle=\color[RGB]{255,150,75}
}
\definecolor{lightlightgray}{RGB}{230,230,230}
\newcommand{\code}[1]{\colorbox{lightlightgray}{\lstinline[language=C]$#1$}}

\section{Source Code and Solutions}

\section{Variables and Types}
\begin{frame}{Integers}
	\begin{itemize}
        \item Keywords: \code{int}, \code{short}, \code{long}, \code{long long}
		\item Stored as a binary number with fixed length
        \item Can be \code{signed} or \code{unsigned} (undefined, but can be overridden using signed char and unsigned char)
		\item Actual size of \code{int}, \code{short}, \code{long} depends on architecture
        \item For definite sizes: include stddef.h which adds types like \code{size_t}, \code{int32t}, \code{uint64_t}
	\end{itemize}
\end{frame}
\begin{frame}[fragile]{float}
\begin{itemize}
		\item Keywords: \code{float}, \code{double}, \code{long double}
		\item Stored as specified in \textit{IEEE 754 Standard} TL;DR
		\item Special values for $\infty$, $-\infty$, NaN
		\item Useful for fractions and very large numbers
		\item Type a decimal point instead of a comma!
	\end{itemize}\ \\
	\ \\
	Example:
	\begin{lstlisting}[numbers=none]
float x = 0.125;			/* Precision: 7 to 8 digits */
double y = 111111.111111;	/* Precision: 15 to 16 digits */
\end{lstlisting}
\end{frame}
\begin{frame}[fragile]{Characters}
	\begin{itemize}
		\item Keyword: \code{char}
		\item Can be \code{signed}(default) or \code{unsigned}	
		\item Size: 1 Byte (8 Bit) on almost every architecture
		\item Intended to represent a single character
		\item Stores its \textit{ASCII} number (e.g. 'A' $\Rightarrow$ 65)
	\end{itemize}\ \\
	\ \\
	You can define a \code{char} either by its ASCII number or by its symbol:
	\begin{lstlisting}[numbers=none]
char a = 65;
char b = 'A';	/* use single quotation marks */
\end{lstlisting}
\end{frame}


\section{Controlstructures}

\subsection{if}

\subsection{switch case}

\subsection{while}

\subsection{do \ldots while}

\subsection{for loop}

\subsection{operators}

\section{format strings}

\section{functions}
\end{document}
