% document
\documentclass[10pt,graphics,aspectratio=169,table]{beamer}
\usepackage{../code}
\usepackage{csquotes}
\usepackage{hyperref}
\usepackage{tikz}
\usepackage{pgfplots}
% theme
\usetikzlibrary{arrows}
\tikzstyle{line}=[draw] % here
\usetikzlibrary{decorations.pathmorphing}
\tikzset{arrow/.style={-latex, shorten >=.5ex, shorten <=.5ex}}

\usetheme{metropolis}
% packages
\title{Lesson 4}
\author{Christian Schwarz, Jakob Krebs}
\date{18.11.2019}
\begin{document}
\maketitle

\begin{frame}{Contents}
    \tableofcontents
\end{frame}


\section{Source Code and Solutions}
\begin{frame}{Sources and Solutions}
    \begin{itemize}
        \item we publish all code written in this course at \url{https://github.com/jkrbs/c_lessons}
        \item we will publish example solutions of the tasks on same site
        \item send us questions or your solutions to c-lessons@deutschland.gmbh
    \end{itemize}
\end{frame}

\section{dynamic memory}
\begin{frame}[fragile]{A closer look at memory}
	\begin{tikzpicture}[y=2cm, font=\footnotesize]
		\node[above, font=\small] at (1.5,0) {Stack};
		
		\draw (0,0) -- (0,-2);
		\draw (0,0) -- (3,0);
		\draw (3,0) -- (3,-2);
		\draw[line join=round, decoration={zigzag,segment length=4, amplitude=1}, decorate] (0,-2) -- (3,-2);
		
		\draw[->] (-.25,0) -- (-.25,-1.9);
		
		\draw[dashed] (0,-.25) -- (3,-.25);
		\draw[dashed] (0,-.5) -- (3,-.5);
		\draw[dashed] (0,-.75) -- (3,-.75);
		\draw[dashed] (0,-1) -- (3,-1);
		\draw[dashed] (0,-1.25) -- (3,-1.25);
		\draw[dashed] (0,-1.5) -- (3,-1.5);
		\draw[dashed] (0,-1.75) -- (3,-1.75);
		
		\node[below, font=\small] at (6.5,-2) {Heap};		
		
		\draw (5,0) -- (5,-2);
		\draw[line join=round, decoration={zigzag,segment length=4, amplitude=1}, decorate] (5,0) -- (8,0);
		\draw (8,0) -- (8,-2);
		\draw (5,-2) -- (8,-2);
		
		\draw[<-] (8.25,-.1) -- (8.25,-2);
		
		\draw[dashed] (5,-.25) -- (8,-.25);
		\draw[dashed] (5,-.5) -- (8,-.5);
		\draw[dashed] (5,-.75) -- (8,-.75);
		\draw[dashed] (5,-1) -- (8,-1);
		\draw[dashed] (5,-1.25) -- (8,-1.25);
		\draw[dashed] (5,-1.5) -- (8,-1.5);
		\draw[dashed] (5,-1.75) -- (8,-1.75);
		
		\draw[thick, dashed, orange] (0,0) -- (3,0);
		\draw[thick, dashed, orange] (0,0) -- (0,-.25);
		\draw[thick, dashed, orange] (0,-.25) -- (3,-.25);
		\draw[thick, dashed, orange] (3,0) -- (3,-.25);
		\node[orange,below=.15em] at (1.5,0){int};

		\draw[thick, dashed, orange] (0,-.25) -- (3,-.25);
		\draw[thick, dashed, orange] (0,-.25) -- (0,-.5);
		\draw[thick, dashed, orange] (0,-.5) -- (3,-.5);
		\draw[thick, dashed, orange] (3,-.25) -- (3,-.5);
		\node[orange,below=.15em] at (1.5,-.25){int};
		
		\begin{uncoverenv}<3->
			\draw[thick, dashed, orange] (5,-1.75) -- (8,-1.75);
			\draw[thick, dashed, orange] (5,-1.75) -- (5,-2);
			\draw[thick, dashed, orange] (5,-2) -- (8,-2);
			\draw[thick, dashed, orange] (8,-1.75) -- (8,-2);
			\node[orange,below=.15em] at (6.5,-1.75){int};
		\end{uncoverenv}
		
		\begin{uncoverenv}<4->
			\draw[thick, dashed, teal] (0,-.5) -- (3,-.5);
			\draw[thick, dashed, teal] (0,-.5) -- (0,-1);
			\draw[thick, dashed, teal] (0,-1) -- (3,-1);
			\draw[thick, dashed, teal] (3,-.5) -- (3,-1);
			\node[teal,below=.15em] at (1.5,-.5){int*};
			
			\draw (3,-.75) edge[out=0,in=180,->,shorten >=.5ex, shorten <=.5ex] (5,-1.875);
		\end{uncoverenv}
		
		\begin{uncoverenv}<5->
			\draw[thick, dashed, purple] (5,-1.25) -- (8,-1.25);
			\draw[thick, dashed, purple] (5,-1.25) -- (5,-1.75);
			\draw[thick, dashed, purple] (5,-1.75) -- (8,-1.75);
			\draw[thick, dashed, purple] (8,-1.25) -- (8,-1.75);
			\node[purple, below=.15em] at (6.5,-1.25){long};
			
			\draw[thick, dashed, teal] (0,-1) -- (3,-1);
			\draw[thick, dashed, teal] (0,-1) -- (0,-1.5);
			\draw[thick, dashed, teal] (0,-1.5) -- (3,-1.5);
			\draw[thick, dashed, teal] (3,-1) -- (3,-1.5);
			\node[teal,below=.15em] at (1.5,-1){long*};
			
			\draw (3,-1.25) edge[out=0,in=180,->,shorten >=.5ex, shorten <=.5ex] (5,-1.5);
		\end{uncoverenv}
	\end{tikzpicture} \\
	\only<1>{All local variables of functions are placed at the \textit{stack}. \\
		It grows and shrinks as variables are declared and functions return.}
	\only<2>{Dynamical memory is allocated on the \textit{heap}. \\
		The example shows a function with two local \textit{int} variables.}
	\begin{onlyenv}<3>
		\begin{codeblock}[numbers=none]
malloc(sizeof(int));
\end{codeblock}
	Reserves exactly the amount of memory an \textit{int} variable takes.
	\end{onlyenv}
	\begin{onlyenv}<4>
		\begin{codeblock}[numbers=none]
int *new_block = malloc(sizeof(int));
\end{codeblock}
	The adress of that memory block is stored in an \textit{int} pointer.
	\end{onlyenv}
	\only<5>{\textit{malloc()} just needs to know the size of the block it reserves. \\
		Let us allocate a \textit{long} variable as well.}
\end{frame}


\begin{frame}[fragile]{\textit{malloc()} in detail}
	The function declaration might be a little bit confusing:
	\begin{codeblock}
void *malloc(size_t size);
    \end{codeblock}
	\begin{itemize}
		\item \textit{size} is the size of the reserved block in \textbf{bytes}. \\
		If you want to use that block \textit{seriously}, pass the size of an actual type (e.g. \textit{sizeof(int)}).
		\item A \textit{void} pointer is returned since \textit{malloc()} does not know how you want to use the reserved block. By assigning it to a regular pointer variable it is automatically converted to that type.
	\end{itemize}
\end{frame}
\begin{frame}[fragile]{Tidying up}
	Unlike normally declared variables, dynamically allocated storage is not automatically released when the function returns.
	\begin{codeblock}
void foo(void) {
	int *bar = malloc(sizeof *bar);
}
\end{codeblock}
	
With the pointer \textit{bar} being removed from the stack, we havo no reference on its allocated memory and those four bytes are blocked forever! \\
	\ \\
	\begin{codeblock}
free(void *ptr);
\end{codeblock}

Pass any pointer to previously allocated memory to \textit{free()} and it gets realeased.
\end{frame}


\end{document}
