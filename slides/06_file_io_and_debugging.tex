% document
\documentclass[10pt,graphics,aspectratio=169,table]{beamer}
\usepackage{../code}
\usepackage{csquotes}
\usepackage{hyperref}
\usepackage{tikz}
\usepackage{pgfplots}
% theme
\usetikzlibrary{arrows}
\tikzstyle{line}=[draw] % here
\usetikzlibrary{decorations.pathmorphing}
\tikzset{arrow/.style={-latex, shorten >=.5ex, shorten <=.5ex}}

\usetheme{metropolis}
% packages
\title{Lesson 6}
\author{Christian Schwarz, Jakob Krebs}
\date{25.11.2019}
\begin{document}
\maketitle

\begin{frame}{Contents}
    \tableofcontents
\end{frame}


\section{File IO and Debugging}
\begin{frame}{Sources and Solutions}
    \begin{itemize}
        \item we publish all code written in this course at \url{https://github.com/jkrbs/c_lessons}
        \item we will publish example solutions of the tasks on same site
        \item send us questions or your solutions to c-lessons@deutschland.gmbh
    \end{itemize}
\end{frame}

\begin{frame}[fragile]{fopen and fclose}
    \code{stdio.h} provides the following functions to open and close a file:

    \begin{codeblock}
FILE* fopen (char* filename, char* mode);
int fclose (FILE* stream);

//example
FILE* test = fopen("test.txt", "w");
fclose(test);
    \end{codeblock}

    Filenames can either be absolute (\code{"/home/foo/bar.txt"}) 
    or relative (\code{"test.txt"}). 
    Relative paths are relative to the "current working directory".
    That is the current directory of your shell when you execute the program.
    Shells can usually change this directory using \code{cd} (change directory), 
    and display it using \code{pwd} (print working directory).


    \textbf{This is not necessarily the directory that the program executable lies in.}  
\end{frame}
    

\begin{frame}{file modes}
    The \code {"w"} mode in \code{fopen(filename, mode)} specifies that we
    only want to \textbf{w}rite to the file.
    There are multiple different modes available: 


    \begin{tabular}{|l|l|l|l|}
        \hline 
        \textbf{mode} & \textbf{access} & \textbf{if file exists} & \textbf{if file doesn't exist}\\ \hline 
        \textbf{r} & read-only & read from start & return NULL \\\hline
        \textbf{w} & write-only & overwrite contents & create new \\\hline
        \textbf{a} & write-only & append & create new \\\hline
        \textbf{r+} & read+write & read from start, overwrite & return NULL \\\hline
        \textbf{w+} & read+write & read from start, overwrite & create new \\\hline
        \textbf{a+} & read+write & read from start, but append at the end& create new \\\hline
    \end{tabular}
\end{frame}

\begin{frame}[fragile]{"File"}
    
    \code{FILE*} can be thought of as a pointer to a FILE structure managed
    by the C standard library that remembers all the necessary information
    to interact with the file. 
    
    "File" should not be taken too literally here.
    Stream might have been the better term.
    For example, \code{stdin} and \code{stdout} are also \code{FILE*}s.

    It really just means an object that bytes 
    can be written to and / or read from.

\end{frame}

\begin{frame}[fragile]{fread and fwrite}
    \begin{small}

    \begin{codeblock}[numbers=none, basicstyle=\small]
size_t fread(void* buffer, size_t size, size_t count, FILE* stream);
size_t fwrite(void* buffer, size_t size, size_t count, FILE* stream);
    \end{codeblock}

    \begin{itemize}
        \item \code{fread} reads bytes from the \code{stream} and writes them into 
        \code{buffer}.  

        \item \code{fwrite} reads bytes from \code{buffer} and writes them out to the 
        \code{stream}. 
    
    \end{itemize}

    The functions read/write \code{size} bytes for up to \code{count} times, or until 
    the stream has no more contents.

    They return the number of elements 
    (of size \code{size}) successfully read/written.

    Sometimes this is useful, e.g. if we want to read up to 20 \code{int}s:
    \begin{codeblock}[numbers=none, basicstyle=\small]
size_t ints_read = fread(buffer, sizeof(int), 20, file);
    \end{codeblock}

    But mostly we use them like this:
    \begin{codeblock}[numbers=none, basicstyle=\small]
size_t bytes_read = fread(buffer, 1, sizeof(buffer), file);
    \end{codeblock}

    \end{small}
\end{frame}

\begin{frame}[fragile]{file io example}
    
    \begin{codeblock}
FILE* config = fopen("config.txt"); 

    \end{codeblock}

\end{frame}

\begin{frame}{useful gdb commands}
    \begin{tabular}{|l|l|}
        \hline
        \textbf{file} & load program\\\hline
        \textbf{r[un]} & execute program\\\hline
        \textbf{b[reak]} & set breakpoint\\\hline
        \textbf{sta[rt]} & execute program and break immediately\\\hline
        \textbf{p[rint]} & print variable\\\hline
        \textbf{w[atch]} & break and print variable when it changes\\\hline
        \textbf{n[ext]} & execute next line and break\\\hline
        \textbf{s[tep]} & execute next instruction and break\\\hline
        \textbf{c[ontinue]} & execute until next breakpoint\\\hline
        \textbf{backtrace} / \textbf{bt} & How did I end up here?\\\hline
    \end{tabular}
\end{frame}

\end{document}
