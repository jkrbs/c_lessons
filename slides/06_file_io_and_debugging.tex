% document
\documentclass[10pt,graphics,aspectratio=169,table]{beamer}
\usepackage{../code}
\usepackage{csquotes}
\usepackage{hyperref}
\usepackage{tikz}
\usepackage{pgfplots}
% theme
\usetikzlibrary{arrows}
\tikzstyle{line}=[draw] % here
\usetikzlibrary{decorations.pathmorphing}
\tikzset{arrow/.style={-latex, shorten >=.5ex, shorten <=.5ex}}

\usetheme{metropolis}
% packages
\title{Lesson 6}
\author{Christian Schwarz, Jakob Krebs}
\date{25.11.2019}
\begin{document}
\maketitle

\begin{frame}{Contents}
    \tableofcontents
\end{frame}


\section{File IO and Debugging}
\begin{frame}{Sources and Solutions}
    \begin{itemize}
        \item we publish all code written in this course at \url{https://github.com/jkrbs/c_lessons}
        \item we will publish example solutions of the tasks on same site
        \item send us questions or your solutions to c-lessons@deutschland.gmbh
    \end{itemize}
\end{frame}

\begin{frame}[fragile]{fopen and fclose}
    To read and write files in c, \code{stdio.h}
    provides the following functions.

    \begin{codeblock}
FILE* fopen (char* filename, char* mode);
int fclose (FILE* stream);
    \end{codeblock}

    With these functions, we can open and close a file. 
    Example:

    \begin{codeblock}
FILE* test = fopen("test.txt", "w");
assert(test != NULL);
//use the file, more on this afterwards
fclose(test);
    \end{codeblock}
\end{frame}
    

\begin{frame}{file modes}
    \code{FILE*} can be thought of as a pointer to a FILE structure managed
    by the C standard library that remembers all the necessary information
    to interact with the file. 

    The \code {"w"} mode in \code{fopen(filename, mode)} specifies that we
    only want to \textbf{w}rite to the file.
    There are multiple different modes available: 


    \begin{tabular}{|l|l|l|l|}
        \hline 
        \textbf{mode} & \textbf{access} & \textbf{if file exists} & \textbf{if file doesn't exist}\\ \hline 
        \textbf{r} & read-only & read from start & return NULL \\\hline
        \textbf{w} & write-only & overwrite contents & create new \\\hline
        \textbf{a} & write-only & append & create new \\\hline
        \textbf{r+} & read+write & read from start, overwrite & return NULL \\\hline
        \textbf{w+} & read+write & read from start, overwrite & create new \\\hline
        \textbf{a+} & read+write & read from start, but append at the end& create new \\\hline
    \end{tabular}
\end{frame}

\begin{frame}{useful gdb commands}
    \begin{tabular}{|l|l|}
        \hline
        \textbf{file} & load program\\\hline
        \textbf{r[un]} & execute program\\\hline
        \textbf{b[reak]} & set breakpoint\\\hline
        \textbf{sta[rt]} & execute program and break immediately\\\hline
        \textbf{p[rint]} & print variable\\\hline
        \textbf{w[atch]} & break and print variable when it changes\\\hline
        \textbf{n[ext]} & execute next line and break\\\hline
        \textbf{s[tep]} & execute next instruction and break\\\hline
        \textbf{c[ontinue]} & execute until next breakpoint\\\hline
        \textbf{backtrace} / \textbf{bt} & How did I end up here?\\\hline
    \end{tabular}
\end{frame}

\end{document}
