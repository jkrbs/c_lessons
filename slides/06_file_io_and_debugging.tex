% document
\documentclass[10pt,graphics,aspectratio=169,table]{beamer}
\usepackage{../code}
\usepackage{csquotes}
\usepackage{hyperref}
\usepackage{tikz}
\usepackage{pgfplots}
% theme
\usetikzlibrary{arrows}
\tikzstyle{line}=[draw] % here
\usetikzlibrary{decorations.pathmorphing}
\tikzset{arrow/.style={-latex, shorten >=.5ex, shorten <=.5ex}}

\usetheme{metropolis}
% packages
\title{Lesson 6}
\author{Christian Schwarz, Jakob Krebs}
\date{02.12.2019}
\begin{document}
\maketitle

\begin{frame}{Contents}
    \tableofcontents
\end{frame}


\section{File IO and Debugging}
\begin{frame}{Sources and Solutions}
    \begin{itemize}
        \item we publish all code written in this course at \url{https://github.com/jkrbs/c_lessons}
        \item we will publish example solutions of the tasks on same site
        \item send us questions or your solutions to c-lessons@deutschland.gmbh
    \end{itemize}
\end{frame}

\begin{frame}[fragile]{fopen and fclose}
    \code{stdio.h} provides the following functions to open and close a file:

    \begin{codeblock}
FILE* fopen (char* filename, char* mode);
int fclose (FILE* stream);

//example
FILE* test = fopen("test.txt", "w");
fclose(test);
    \end{codeblock}

    Filenames can either be absolute (\code{"/home/foo/bar.txt"}) 
    or relative (\code{"test.txt"}). 
    Relative paths are relative to the "current working directory".
    That is the current directory of your shell when you execute the program.
    Shells can usually change this directory using \code{cd} (change directory), 
    and display it using \code{pwd} (print working directory).


    \textbf{This is not necessarily the directory that the program executable lies in.}  
\end{frame}
    

\begin{frame}{file modes}
    The \code {"w"} mode in \code{fopen(filename, mode)} specifies that we
    only want to \textbf{w}rite to the file.
    There are multiple different modes available: 


    \begin{tabular}{|l|l|l|l|}
        \hline 
        \textbf{mode} & \textbf{access} & \textbf{if file exists} & \textbf{if file doesn't exist}\\ \hline 
        \textbf{r} & read-only & read from start & return NULL \\\hline
        \textbf{w} & write-only & overwrite contents & create new \\\hline
        \textbf{a} & write-only & append & create new \\\hline
        \textbf{r+} & read+write & read from start, overwrite & return NULL \\\hline
        \textbf{w+} & read+write & read from start, overwrite & create new \\\hline
        \textbf{a+} & read+write & read from start, but append at the end& create new \\\hline
    \end{tabular}
\end{frame}

\begin{frame}[fragile]{"File"}
    
    \code{FILE*} can be thought of as a pointer to a FILE structure managed
    by the C standard library that remembers all the necessary information
    to interact with the file. 
    
    "File" should not be taken too literally here.
    Stream might have been the better term.
    For example, \code{stdin} and \code{stdout} are also \code{FILE*}s.

    It really just means an object that bytes 
    can be written to and / or read from.

\end{frame}

\begin{frame}[fragile]{fread and fwrite}
    \begin{small}

    \begin{codeblock}[numbers=none, basicstyle=\small]
size_t fread(void* buffer, size_t size, size_t count, FILE* stream);
size_t fwrite(void* buffer, size_t size, size_t count, FILE* stream);
    \end{codeblock}

    \begin{itemize}
        \item \code{fread} reads bytes from the \code{stream} and writes them into 
        \code{buffer}.  

        \item \code{fwrite} reads bytes from \code{buffer} and writes them out to the 
        \code{stream}. 
    
    \end{itemize}

    The functions read/write \code{size} bytes for up to \code{count} times, or until 
    the stream has no more contents.

    They return the number of elements 
    (of size \code{size}) successfully read/written.

    Sometimes this is useful, e.g. if we want to read up to 20 \code{int}s:
    \begin{codeblock}[numbers=none, basicstyle=\small]
size_t ints_read = fread(buffer, sizeof(int), 20, file);
    \end{codeblock}

    But mostly we use them like this:
    \begin{codeblock}[numbers=none, basicstyle=\small]
size_t bytes_read = fread(buffer, 1, sizeof(buffer), file);
    \end{codeblock}

    \end{small}
\end{frame}

\begin{frame}[fragile]{file io example}
    
    \begin{codeblock}
FILE* logfile = fopen("log.txt", "a+");
// very unlikely to fail since "a+" creates nonexistant files
assert(logfile != NULL);

char buffer[1024]; 
do{
    size_t size = fread(&buffer, 1, sizeof(buffer), logfile);
    display_log(&buffer, size); // use the data
} while(bytes > 0);

char* msg = "we accessed the log file\n";
size_t size = fwrite(msg, strlen(msg), 1, log);
assert(size == 1); // was our data written successfully ?

fclose(config);
    \end{codeblock}

\end{frame}
\section{debugging}
\begin{frame}{Motivation}
we make mistakes in our code and want to see the state of our program (variables) at a certain state

There are different kinds of errors.
	\begin{itemize}
		\item Compiletime errors
		\item Runtime errors (\textit{bugs})
	\end{itemize}\ \\\ \\
	\textit{Compiletime errors} are easily handable since the compiler shows you where to fix them.\\\ \\
	\textit{Bugs} on the other hand are harder to find because you have no idea where to look for them.
\end{frame}

\begin{frame}{bug types}
	Bugs can appear due to different reasons
	\begin{itemize}
		\item Variable overflow
		\item Division by zero
		\item Infinite loops / recursions
		\item Range excess
		\item Segmentation fault
		\item Dereferencing \textit{NULL pointers}
		\item ...
	\end{itemize}
\end{frame}

\begin{frame}{debuggers}

Debuggers help us to find those issues, e.g. by alllowing use to pause the program at a certain line 
most development envirements have integrated debugging systems.

popular tools are:

\begin{itemize}
    \item gdb (GNU Debugger) for general debuging
    \item valgrind for finding memory leaks and other memory issues
\end{itemize}
\end{frame}
\begin{frame}[fragile]{The \textbf{G}NU \textbf{D}e\textbf{B}ugger}
	There are tools helping with bugs, called debuggers. GDB is one of them.\\
	\bigskip
	To use it
	\begin{itemize}
		\item You have to install the package \textit{gdb}\\
		\item You have to compile your program with the \textit{-g} flag
\begin{code}
gcc -g main.c
\end{code}
		\item After that you can start your program with gdb:
		\begin{code}
gdb a.out
\end{code}
	\end{itemize}
\end{frame}
\begin{frame}{Commands}
	\begin{itemize}
		\item If you started gdb without a file you can load it with \textbf{file} \textit{file\_name}.
		\item Use \textbf{r[un]} to execute the program with gdb.\\
		You should begin with that. It will give you further information about the crash.
		\item You can set an arbitrary amount of breakpoints with \textbf{b[reak]} \textit{line\_number} or \textbf{b[reak]} \textit{function\_name}.\\
		Begin with a breakpoint at the point the program crashes.
		\item Print values with \textbf{p[rint]} \textit{identifier}.
		\item Use \textbf{w[atch]} \textit{identifier} to break and print a variable when it's changed.
	\end{itemize}
\end{frame}
\begin{frame}{Once you're at a breakpoint}
	\begin{itemize}
		\item Use \textbf{n[ext]} to execute the next program line only.
		\item \textbf{s[tep]} executes the next instruction.
		\item You can jump to the next breakpoint with \textbf{c[ontinue]}.
		\item To see how you have come to this point in the program flow, type \textbf{backtrace} or \textbf{bt}.\\
		This shows you all functions you called to come there.
		\\\ \\
		\item By only hitting the \textit{return key}, you repeat the last entered  command.
	\end{itemize}
	
\end{frame}
\begin{frame}[fragile]{Conditional breakpoints}
After setting a breakpoint, GDB assigns an ID to it.\\
You can use this ID to extend the functionality of that breakpoint.
	\begin{itemize}
		\item \textbf{con[dition]} \textit{breakpoint\_ID expression} adds a condition to your Breakpoint:
		\begin{lstlisting}[numbers=none,language=bash]
(gdb) br 42
Breakpoint 1 at 0xbada55: file main.c, line 42.
(gdb) condition 1 i@=@@=@3
\end{lstlisting}
		\item For string comparison, set the string before comparing with \textbf{strcmp}:
		\begin{code}
(gdb) br main.c:42
Breakpoint 13 at 0xdeadbeef: file main.c, line 42.
(gdb) set $string_to_compare = "lolwut"
(gdb) cond 13 strcmp ( $stringtocompare, c ) @=@@=@ 0
\end{code}
	\end{itemize}
\end{frame}
\begin{frame}{The dungeon}
	We copied a little ASCII dungeon from former c courses.\\
	You can find it in the repository (\url{https://jkrbs.github.io/c_lessons/src/dungeon.c}\\
	\begin{itemize}
		\item Look at the code and try to understand what should happen.
		\item If you find mistakes, please leave them. We'll fix them later.
		\\\ 
		\item Compile it (with \textit{-std=c99}).
		\\\ 
		\item And now run it.
	\end{itemize}
\end{frame}
\begin{frame}{useful gdb commands}
    \begin{tabular}{|l|l|}
        \hline
        \textbf{file} & load program\\\hline
        \textbf{r[un]} & execute program\\\hline
        \textbf{b[reak]} & set breakpoint\\\hline
        \textbf{sta[rt]} & execute program and break immediately\\\hline
        \textbf{p[rint]} & print variable\\\hline
        \textbf{w[atch]} & break and print variable when it changes\\\hline
        \textbf{n[ext]} & execute next line and break\\\hline
        \textbf{s[tep]} & execute next instruction and break\\\hline
        \textbf{c[ontinue]} & execute until next breakpoint\\\hline
        \textbf{backtrace} / \textbf{bt} & How did I end up here?\\\hline
    \end{tabular}
\end{frame}

\end{document}
