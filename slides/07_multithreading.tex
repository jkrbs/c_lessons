% document
\documentclass[10pt,graphics,aspectratio=169,table]{beamer}
\usepackage{../code}
\usepackage{csquotes}
\usepackage{hyperref}
\usepackage{tikz}
\usepackage{pgfplots}
% theme
\usetikzlibrary{arrows}
\tikzstyle{line}=[draw] % here
\usetikzlibrary{decorations.pathmorphing}
\tikzset{arrow/.style={-latex, shorten >=.5ex, shorten <=.5ex}}

\usetheme{metropolis}
% packages
\title{Lesson 7}
\author{Christian Schwarz, Jakob Krebs}
\date{9.12.2019}
\begin{document}
\maketitle

\begin{frame}{Contents}
    \tableofcontents
\end{frame}

\begin{frame}{Sources and Solutions}
    \begin{itemize}
        \item we publish all code written in this course at \url{https://github.com/jkrbs/c_lessons}
        \item we will publish example solutions of the tasks on same site
        \item send us questions or your solutions to c-lessons@deutschland.gmbh
    \end{itemize}
\end{frame}


\section{Function Pointers}
\begin{frame}[fragile]{Function Pointers}
    \begin{itemize}
        \item As we know, a pointer is really just a memory address
        \item The code for functions is also in memory, so it also has an address.
        \item As long as different functions have the same argument types, we
        can call them using a function pointer:
    \end{itemize}

    \begin{codeblock}
int foo(int x, int y){ puts("called foo"); }
int foo(int x, int y){ puts("called bar"); }
// func_ptr is the newly created function pointer 
int (*func_ptr)(int,int);
func_ptr = &foo; // we give it foo's address
func_ptr(1,2); // we call foo
func_ptr = bar; // the '&' is optional, think arrays
func_ptr(1,2); // we call bar
    \end{codeblock}

\end{frame}


\begin{frame}[fragile]{Function Pointer Typedefs}
    Since the above type declaration is slighly complicated,
    it's often a good idea to alias the function type:
    \begin{codeblock}
// we can give the parameters names but it's optional
typedef int (*event_callback)(int event_id, void* context);

// now we can initialize a variable like this:
event_callback callback = &my_event_handler;
int result = callback(1, NULL);
    \end{codeblock}

\end{frame}

\section{Type Qualifiers}
\begin{frame}[fragile]{const}
    To give more information about a variable to the compiler, you can\\
    \textit{qualify} its type. 
    
    The most common type qualifier is \code{const}.
    It prevents the qualified variable from being modified.
    If you try anyways you will get a compiler error.

    \begin{codeblock}
// request that x can't be written to (after initialization)
int const x = 3; 
x = 3; // error: assignment of read-only variable 'x'

void f(int *a); // forward declaration for f
f(&i); // warning: 'foo' [...] discards 'const' qualifier [...]
    \end{codeblock}

    But this is C we're talking about, so of course there's a way: 
    \code{*(int*)&x = 3;}

    Compiles no problem. \textbf{But} what happens is undefined behaviour,
    so your progamm would no longer be valid.

\end{frame}

\begin{frame}[fragile]{From the west-const to the east-const}
    \bigskip
    Normally, a qualifier refers to the type to its \textbf{left},
    but the following is also valid (and more common!):
    \begin{codeblock}
const int a;	        // equal to `int const a`
    \end{codeblock}

    Watch complex types:
    
    \begin{codeblock}
const int *foo;         // mutable pointer, constant integer
int const *foo;	        // same as above
int * const foo;        // constant pointer, mutable integer 
int const * const foo;  // everything constant 
    \end{codeblock}
    \end{frame}

    \begin{frame}[fragile]{volatile}
 \code{volatile} prevents the compiler from doing aggressive
                optimizations on a variable. For example:

        \begin{codeblock}
volatile bool interrupt_occured = false;
while(!interrupt_occured){
    // we don't change interrupt_occured, so the compiler
    // might assume that it can optimize away the check
}
        \end{codeblock}

        This is mainly used in low-level programming:
        \begin{itemize}
            \item Hardware access (memory-mapped I/O)
            \item Threading (another thread modifies a value) (\textbf{!!} be very careful here)
        \end{itemize}
    \end{frame}

    \begin{frame}[fragile]{volatile example C}
        
        \begin{columns}
        \column{.5\textwidth}
        \begin{codeblock}
#include <stdio.h>

int main(void) {
    int i = 42;
    printf("%d\n", i);
}
        \end{codeblock}
        \column{.5\textwidth}
        \begin{codeblock}
#include <stdio.h>

int main(void) {
    volatile int i = 42;
    printf("%d\n", i);
}
        \end{codeblock}
        \end{columns}
    \end{frame}
    \begin{frame}[fragile]{volatile example assembly}
        After compilation with \textit{gcc -O3}:
        \begin{columns}
        \column{.5\textwidth}
            \begin{small}
            \begin{lstlisting}[escapeinside={(*}{*)}]
.LC0:
    .string "%d\n"
main:
    sub     rsp, 8
    mov     esi, 42
    mov     edi, OFFSET FLAT:.LC0
    xor     eax, eax
    call    printf
    xor     eax, eax
    add     rsp, 8
    ret
            \end{lstlisting}
            \end{small}
        \column{.5\textwidth}
            \begin{small}
            \begin{lstlisting}[escapeinside={(*}{*)}]
.LC0:
    .string "%d\n"
main:
    sub     rsp, 24
    mov     edi, OFFSET FLAT:.LC0
    xor     eax, eax
    mov     DWORD PTR [rsp+12], 42
    mov     esi, DWORD PTR [rsp+12]
    call    printf
    xor     eax, eax
    add     rsp, 24
    ret
            \end{lstlisting}
            \end{small}
        \end{columns}
        \bigskip
        The compiler could not pass 42 to \code{printf} directly once we 
        made \code{i} \code{volatile}.
        \end{frame}

    \begin{frame}[fragile]{restrict}
        Restrict guarantees to the compiler that nobody else is 
        writing to the memory of a pointer (the pointer is not aliased).
Therefore the compiler might do an optimization like this:
        \begin{codeblock}
void f(char *restrict p1, char *restrict p2) {
    for (int i = 0; i < 50; i++) {
        p1[i] = 4;
        p2[i] = 9;
    }
}
// optimized version, only valid if p1 and p2 don't overlap
void f(char *restrict p1, char *restrict p2) {
    memset(p1, 4, 50);
    memset(p2, 9, 50);
}
    \end{codeblock}

    Since this is purely an optimization, \code{restrict} never changes the output
    of a valid program.
    \end{frame}

\section{Parallelism}
\subsection{}
\begin{frame}{Executing code in parallel}
    Each program has a process associated with it. At program start, this process has
    exactly one thread executing your \code{main} function.\\
    \bigskip
    To achieve parallelism, you can
    \begin{itemize}
            \item create a new process running the same code
            \item call a function in a new thread
    \end{itemize}
    \bigskip
    In Unix systems, processes are created with the \code{fork} system call.\\
    The new process will have its own memory to work with.\\
    For starting threads, libraries such as \code{p[osix]threads} are used.\\
    All threads of a process share the same memory.
\end{frame}

\begin{frame}[fragile]{Use the fork}
    \begin{codeblock}
#include <unistd.h>
int main(void) {
    pid_t pid = fork();
    if (pid == 0) {
        /* do stuff in child process */
    } else if (pid > 0) {
        /* do stuff in parent process */
    } else {
        /* fork failed */
        return 1;
    }
    return 0;
}
\end{codeblock}

    Have a look at \code{man 2 fork} for further information.
\end{frame}

\section{pthreads}
\subsection{}
\begin{frame}[fragile]{pthread\_create}
    To execute a function in a new thread, use:
    \begin{codeblock}
int pthread_create(pthread_t *thread,
                   const pthread_attr_t *attr,
                   void *(*start_routine) (void *),
                   void *arg);
\end{codeblock}

    where
    \begin{itemize}
            \item *\code{thread} is where the thread's id will be stored
            \item *\code{attr} contains attributes for the thread (pass \code{NULL} for default)
            \item \code{start\_routine} is the function to execute. Both the single argument and the return value must be \code{void *}.
            \item \code{arg} is passed to the function to be used as an argument
    \end{itemize}
\end{frame}

\begin{frame}[fragile]{code example: threads}
    \begin{codeblock}
#include <pthread.h>
#include <stdio.h>

void *hello_thread(void *tid) {
    printf("Hello, I am thread %d\n", *(int*) tid);
    pthread_exit(NULL);
}

int main(void) {
    pthread_t threads[5];
    for (int i=0; i < 5; ++i) {
       if (pthread_create(&threads[i], NULL,
                          hello_thread, (void *) i))
          return 1;
    }
    return 0;
}

\end{codeblock}
\end{frame}
\begin{frame}[fragile]{How threads end}
    \begin{itemize}
        \item \code{pthread\_exit} is called
        \begin{codeblock}
void pthread_exit(void *retval);
\end{codeblock}
        \item \code{pthread\_cancel} is called from another thread
        \begin{codeblock}
int pthread_cancel(pthread_t thread);
\end{codeblock}
        \item \code{exit} is called from any thread (ending the process)
        \item \code{main} finishes before \code{startdd\_routine} (ending the process)
    \end{itemize}
\end{frame}
\begin{frame}[fragile]{Waiting for threads}
    To wait for a thread to finish, there is \code{pthread\_join}
    \begin{codeblock}
int pthread_join(pthread_t thread, void **retval);
\end{codeblock}
    \bigskip
    The thread passed to \code{pthread\_join} must be joinable. For that we need the
    \code{pthread\_attr\_t} structure. Here is how to initialise and destroy it:\\
        \begin{codeblock}
int pthread_attr_init(pthread_attr_t *attr);
int pthread_attr_destroy(pthread_attr_t *attr);
\end{codeblock}
    \bigskip
    How to set/get the detachstate (pass \code{PTHREAD\_CREATE\_JOINABLE}):
    \begin{codeblock}
int pthread_attr_setdetachstate(pthread_attr_t *attr,
                                int detachstate);
int pthread_attr_getdetachstate(const pthread_attr_t *attr,
                                int *detachstate);
\end{codeblock}
\end{frame}

\begin{frame}[fragile]{code example: joinable threads}
    \begin{codeblock}
int main(void) {
    pthread_t threads[5];
    pthread_attr_t attr;
    
    pthread_attr_init(&attr);
    pthread_attr_setdetachstate(&attr, PTHREAD_CREATE_JOINABLE);
    
    for (int i=0; i < 5; ++i) {
       if (pthread_create(&threads[i], &attr,
                          hello_thread, (void *) i))
          return 1;
    }
    pthread_attr_destroy(&attr);

    void *st;
    for (int i=0; i < 5; ++i) {
       if (pthread_join(thread[i], &st))
           return 1;
       printf("Thread %d finished with %d\n", i, *(int *) st);
    }
    
    return 0;
}
\end{codeblock}
\end{frame}

\section{Mutual Exclusion}
\subsection{}
\begin{frame}[fragile]{Mutexes}
    Threads can communicate with each other by manipulating global variables or the value behind the \code{arg} pointer we pass to \code{pthread\_create}.\\
    To avoid race conditions, the pthread library provides mutexes.
    \begin{codeblock}
int pthread_mutex_destroy(pthread_mutex_t *mutex);
int pthread_mutex_init(pthread_mutex_t *restrict mutex,
                 const pthread_mutexattr_t *restrict attr);
pthread_mutex_t mutex = PTHREAD_MUTEX_INITIALIZER;
\end{codeblock}
    \bigskip
    
    A mutex is a datatype that can be locked before and unlocked after accessing a variable.
    \begin{codeblock}
int pthread_mutex_lock(pthread_mutex_t *mutex);
int pthread_mutex_unlock(pthread_mutex_t *mutex);
\end{codeblock}
\end{frame}

\begin{frame}[fragile]{mutex example}
    \begin{codeblock}
struct stuff {
    unsigned a;
    unsigned b;
}

struct stuff global = {1, 2};
pthread_mutex_t mutex;

void *thread(void *tid) {
    pthread_mutex_lock(mutex);
    global.b = a;
    pthread_mutex_unlock(mutex);
    
    pthread_exit(NULL);
}
\end{codeblock}
\end{frame}

\begin{frame}[fragile]{Deadlocks incoming}
     \begin{codeblock}
void *thread_1(void *tid) {
    pthread_mutex_lock(mutex_1);
    pthread_mutex_lock(mutex_2);
    /* stuff */
    pthread_mutex_unlock(mutex_1);
    pthread_mutex_unlock(mutex_2);
    pthread_exit(NULL);
}
void *thread_2(void *tid) {
    pthread_mutex_lock(mutex_2);
    pthread_mutex_lock(mutex_1);
    /* stuff */
    pthread_mutex_unlock(mutex_2);
    pthread_mutex_unlock(mutex_1);
    pthread_exit(NULL);
}
    \end{codeblock}
\end{frame}

\end{document}
