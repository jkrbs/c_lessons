% document
\documentclass[10pt,graphics,aspectratio=169,table]{beamer}
\usepackage{../code}
\usepackage{csquotes}
\usepackage{hyperref}
\usepackage{tikz}
\usepackage{pgfplots}
% theme
\usetikzlibrary{arrows}
\tikzstyle{line}=[draw] % here
\usetikzlibrary{decorations.pathmorphing}
\tikzset{arrow/.style={-latex, shorten >=.5ex, shorten <=.5ex}}

\usetheme{metropolis}
% packages
\title{Lesson 9}
\author{Christian Schwarz, Jakob Krebs}
\date{15.12.2019}
\begin{document}
\maketitle

\begin{frame}{Contents}
    \tableofcontents
\end{frame}

\begin{frame}{Sources and Solutions}
    \begin{itemize}
        \item we publish all code written in this course at \url{https://github.com/jkrbs/c_lessons}
        \item we will publish example solutions of the tasks on same site
        \item send us questions or your solutions to c-lessons@deutschland.gmbh
    \end{itemize}
\end{frame}

\section{unions}
\begin{frame}[fragile]{Why do we need Unions?}
    Suppose you have a struct representing one of different kinds of 
    events in an game:
    \begin{codeblock}
struct game_event{
    enum event_kind {
        PLAYER_INTERACTION, PROJECTILE_HIT //...
    } kind;
    // only used by PLAYER_INTERACTION events   
    enum button_kind button; 
    // only used by PROJECTILE_HIT events
    float projectile_hit_speed; 
};
    \end{codeblock}

    \begin{itemize}
        \item This struct layout wouldn't be very efficient, since
        \code{PLAYER_INTERACTION} events never need \code{projectile_hit_speed},
        but waste memory for it anyway.
        \item This can become a problem if we have many events or start
        to transmit events over the network in multiplayer games.
    \end{itemize} 
 
\end{frame}

\begin{frame}[fragile]{How can we save memory using unions?}
    We can tell C that we only need one of a list of fields at a 
    time using \code{unions}.

    \begin{codeblock}
struct game_event{
    enum event_kind {PLAYER_INTERACTION, PROJECTILE_HIT} kind;
    union {
        enum button_kind button;    
        float projectile_hit_speed; 
    } payload;
};
    \end{codeblock}
    \begin{itemize}
        \item Syntactically, a union behaves a lot like a struct.
        \item But, C will use the same memory block for all members
        of the \code{union}
        \item Therefore, the size of the \code{union} will 
        be equal to it's biggest member.
        \item Writing to one union member and then reading from another causes
            undefined behavior (Though it's quite commonly done anyways)
    \end{itemize}
\end{frame}

\begin{frame}[fragile]{Anonymous Unions}
    Sometimes we just want to share memory without giving a name to the 
    created substructure. For this, \code{C11} (2011) introduced / standardized 
    anonymous unions and structs:

    \begin{codeblock}
struct game_event{
    enum event_kind {PLAYER_INTERACTION, PROJECTILE_HIT} kind;
    union {
        enum button_kind button;    
        struct {
            int projectile_damage; 
            float projectile_hit_speed; 
        }; //struct members don't share memory inside a union
    };
};
game_event g;
g.projectile_hit_speed = 3; //we can access the members directly
    \end{codeblock}
\end{frame}


\section{procedural macros}

\begin{frame}[fragile]{Procedural Macros}
    So far we only talked about the simplest types of macros: 
    \begin{codeblock}
#define MY_CONSTANT 5
    \end{codeblock}

    But Macros can also be used like functions to and take parameters:
    \begin{codeblock}
#define MULT(a, b) a * b
    \end{codeblock}
    \begin{itemize}
        \item Like with all macros, this is just a textual replacement.
        \item Therefore \code{MULT(1 + 1, 2)} will yield 3,
         not 4, as it expands to \code{1 + 1 * 2}
        \item we can write \code{#define MULT(a, b) ((a) * (b))}
         to avoid such problems
        \item Why not just use a function? 
        \begin{itemize}
            \item C doesn't evaluate function calls
            at compile time, which prevents using the result e.g.\ for array bounds. 
            \item Macros can be used to generate boilerplate code like
                slightly differing structs, functions or constants 
            \item Macro arguments can be (partial) C code!
        \end{itemize}
    \end{itemize}
\end{frame}

\begin{frame}[fragile]{Function like Macros}
    When a macro contains multiple statements but is meant to be called like
    a function with a trailing semicolon and behave as expected e.g.
    when we do \code{if(some_cond) MY_MACRO(3);} then we need a little hack: 
    \begin{codeblock}
#define MY_MACRO(x) do{             \
    foo(x);                         \
    bar();                          \
} while(false) //no semicolon! 
    \end{codeblock}

    \begin{itemize}
        \item The backslashes are necessary to continue the macro line,
            therefore no backslash is necessary on the last line
        \item if we had just put the statements without the do block, 
            an \code{if} would only conditionalize the first statement.
        \item by leaving out the semicolon after the \code{while(false)} 
            we create semantics equal to a (void) function call
        \item It's not possible to 'return' a value using this trick
    \end{itemize}
\end{frame}


\begin{frame}[fragile]{Variadic Macros}
    Macros can also take a variable amount of arguments, the so called
    parameter pack is referenced using the special macro \code{__VA_ARGS__}

    \begin{codeblock}
#define printfln(fmt, ...) printf(fmt "\n", __VA_ARGS__)
    \end{codeblock}
    \begin{itemize}
        \item parameter packs must always contain at least one argument,
        otherwise we might get an error
        \item many compilers allow it anyways or offer tricks to deal with 
            empty parameter packs (e.g. gccs \code{##__VA_ARGS__})
    \end{itemize}
\end{frame}

\section{unix privileges}
\begin{frame}[fragile]{basics}
In unix the file permissions can be set for the user, the group and for all users on the system.

A file can be executable, readable and writable for everyone/group/user.

Each user has a user id \code{uid} and each group has a group id \code{gid}.

\end{frame}

\begin{frame}[fragile]{permission numbers}

\begin{tabular}{l|l}
octal & permission\\
\hline \\
0 & \texttt{---} \\
1 &	\texttt{--x} \\
2 &	\texttt{-w-} \\
3 &	\texttt{-wx} \\
4 &	\texttt{r--} \\
5 &	\texttt{r-x} \\
6 &	\texttt{rw-} \\
7 &	\texttt{rwx} \\
\end{tabular}
easy to calculate \code{4(r) + 1 (x) = 5 (r-x)}

\begin{codeblock}
> ls -lah
-rw-r--r--  1 jakob jakob 2.7K Dec  7 11:49 default.php
drwxr-xr-x  2 jakob jakob 4.0K Dec  5 13:25 dl.txrx23.de/
-rw-------  1 root  root     5 Jan  3 08:42 extnetif.txt
\end{codeblock}

\end{frame}

\begin{frame}[fragile]{setuid and setgid}

With the setuid and setgid bit set the file will always be executed with the permissions
of the owning group or the owning user.

This is indicated by a \code{s} in the user or group executable position

\begin{codeblock}
~/u/c/c/src> ls -lah a.out 
-rwsr-xr-x 1 root root 17K Jan  5 18:33 a.out*
\end{codeblock} 
\end{frame}
\begin{frame}[fragile]{in c}
\begin{codeblock}
#include <unistd.h>

uid_t getuid(void);
uid_t getgid(void);

uid_t geteuid(void);
uid_t getegid(void);

int setuid(uid_t uid);
int setgid(uid_t uid);
int seteuid(uid_t uid);
int setegid(uid_t uid);
\end{codeblock}

The \code{getuid()} function shall return the real user ID  of  the  calling process.

The \code{geteuid()} function shall return the effective user ID.

\end{frame}

\begin{frame}[fragile]{dropping privileges}
we want to run with as little privileges as possible. So after needing privileges for starting our application, we drop as many as possible.

For this we want to set the effective user id of our program to a uid with less privileges

\begin{codeblock}
if (setegid(target_groupid) != 0)
    //err
if (seteuid(target_userid) != 0)
    //err
\end{codeblock}

\end{frame}

\section{project}

\begin{frame}{project}
    We are finished with the content, which is appropriate for a c course. 
So we thought it would be time for a more challenging task.
\begin{itemize}
    \item Do you want to work on the project as a group?
    \item Do you have ideas on a small game we could write?
    \item Shall it be a game or something different?
    \item Should we have small blocks of content at the beginning of the next lessons?
\end{itemize}
Our proposal is a multiplayer rouge-like game. (basically a small ascii dungeon in which two people have to fight monsters)
\end{frame}

\end{document}
