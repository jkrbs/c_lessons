% document
\documentclass[10pt,graphics,aspectratio=169,table]{beamer}
\usepackage{../code}
\usepackage{csquotes}
\usepackage{hyperref}
\usepackage{tikz}
\usepackage{pgfplots}
% theme
\usetikzlibrary{arrows}
\tikzstyle{line}=[draw] % here
\usetikzlibrary{decorations.pathmorphing}
\tikzset{arrow/.style={-latex, shorten >=.5ex, shorten <=.5ex}}

\usetheme{metropolis}
% packages
\title{Lesson 9}
\author{Christian Schwarz, Jakob Krebs}
\date{15.12.2019}
\begin{document}
\maketitle

\begin{frame}{Contents}
    \tableofcontents
\end{frame}

\begin{frame}{Sources and Solutions}
    \begin{itemize}
        \item we publish all code written in this course at \url{https://github.com/jkrbs/c_lessons}
        \item we will publish example solutions of the tasks on same site
        \item send us questions or your solutions to c-lessons@deutschland.gmbh
    \end{itemize}
\end{frame}

\section{unions}
\begin{frame}[fragile]{Why do we need Unions?}
    Suppose you have a struct representing one of different kinds of 
    events in an game:
    \begin{codeblock}
struct game_event{
    enum event_kind {
        PLAYER_INTERACTION, PROJECTILE_HIT //...
    } kind;
    // only used by PLAYER_INTERACTION events   
    enum button_kind button; 
    // only used by PROJECTILE_HIT events
    float projectile_hit_speed; 
};
    \end{codeblock}

    \begin{itemize}
        \item This struct layout wouldn't be very efficient, since
        \code{PLAYER_INTERACTION} events never need \code{projectile_hit_speed},
        but waste memory for it anways.
        \item This can become a problem if we have many events or start
        to transmit events over the network in multiplayer games.
    \end{itemize} 
 
\end{frame}

\begin{frame}[fragile]{How can we save memory using unions?}
    We can tell C that we only need one of a list of fields at a 
    time using \code{unions}.

    \begin{codeblock}
struct game_event{
    enum event_kind {PLAYER_INTERACTION, PROJECTILE_HIT} kind;
    union {
        enum button_kind button;    
        float projectile_hit_speed; 
    } payload;
};
    \end{codeblock}
    \begin{itemize}
        \item Syntactically, a union behaves a lot like a struct.
        \item But, C will use the same memory block for all members
        of the \code{union}
        \item Therefore, the size of the \code{union} will 
        be equal to it's biggest member.
        \item Writing to one union member and then reading from another causes
            undefined behaviour (Though it's quite commonly done anyways)
    \end{itemize}
\end{frame}

\begin{frame}[fragile]{Anonymous Unions}
    Sometimes we just want to share memory without giving a name to the 
    created substructure. For this, \code{C11} (2011) introduced / standardized 
    anonymous unions and structs:

    \begin{codeblock}
struct game_event{
    enum event_kind {PLAYER_INTERACTION, PROJECTILE_HIT} kind;
    union {
        enum button_kind button;    
        struct {
            int projectile_damage; 
            float projectile_hit_speed; 
        }; //struct members dont't share memory inside a union
    };
};
game_event g;
g.projectile_hit_speed = 3; //we can access the members directly
    \end{codeblock}
\end{frame}


\section{procedural macros}

\begin{frame}[fragile]{Procedural Macros}
    So far we only talked about the simplest types of macros: 
    \begin{codeblock}
#define MY_CONSTANT 5
    \end{codeblock}

    But Macros can also be used like functions to and take parameters:
    \begin{codeblock}
#define MULT(a, b) a * b
    \end{codeblock}
    \begin{itemize}
        \item Like with all macros, this is just a textual replacement.
        \item Therefore \code{MULT(1 + 1, 2)} will yield 3,
         not 4, as it exands to \code{1 + 1 * 2}
        \item we can write \code{#define MULT(a, b) ((a) * (b))}
         to avoid such problems
        \item Why not just use a function? 
        \begin{itemize}
            \item C doesn't evaluate function calls
            at compile time, which prevents using the result e.g. for array bounds. 
            \item Macros can be used to generate boilerplate code like
                slightly differing structs, functions or constants 
            \item Macro arguments can be (partial) C code!
        \end{itemize}
    \end{itemize}
\end{frame}

\begin{frame}[fragile]{Function like Macros}
    When a macro contains multiple statements but is meant to be called like
    a function with a trailing semicolon and behave as expected e.g.
    when we do \code{if(some_cond) MY_MACRO(3);} then we need a little hack: 
    \begin{codeblock}
#define MY_MACRO(x) do{             \
    foo(x);                         \
    bar();                          \
} while(false) //no semicolon! 
    \end{codeblock}

    \begin{itemize}
        \item The backslashes are necessary to continue the macro line,
            therefore no backslash is necessary on the last line
        \item if we had just put the statements without the do block, 
            an \code{if} would only conditionalize the first statement.
        \item by leaving out the semicolon after the \code{while(false)} 
            we create semantics equal to a (void) fuction call
        \item It's not possible to 'return' a value using this trick
    \end{itemize}
\end{frame}


\begin{frame}[fragile]{Variadic Macros}
    Macros can also take a variable amount of arguments, the so called
    parameter pack is referenced using the special macro \code{__VA_ARGS__}

    \begin{codeblock}
#define printfln(fmt, ...) printf(fmt "\n", __VA_ARGS__)
    \end{codeblock}
    \begin{itemize}
        \item parameter packs must always contain at least one argumnent,
        otherwise we might get an error
        \item many compilers allow it anyways or offer tricks to deal with 
            empty parameter packs (e.g. gccs \code{##__VA_ARGS__})
    \end{itemize}
\end{frame}

\section{unix privileges}

\end{document}
