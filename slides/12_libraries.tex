% document
\documentclass[10pt,graphics,aspectratio=169,table]{beamer}
\usepackage{../code}
\usepackage{csquotes}
\usepackage{hyperref}
\usepackage{tikz}
\usepackage{pgfplots}
% theme
\usetikzlibrary{arrows}
\tikzstyle{line}=[draw] % here
\usetikzlibrary{decorations.pathmorphing}
\tikzset{arrow/.style={-latex, shorten >=.5ex, shorten <=.5ex}}

\usetheme{metropolis}
% packages
\title{Lesson 12}
\author{Christian Schwarz, Jakob Krebs}
\date{27.1.2020}
\begin{document}
\maketitle

\begin{frame}{Contents}
    \tableofcontents
\end{frame}


\section{Libraries}

\subsection{Static Libraries}
\begin{frame}{Static Libraries}
    \begin{itemize}
        \item Compilation is hard, and can take a long time. 
            The compiler needs to do a lot of work to get 
            from C code to an executable binary.
        \item When our Project becomes large, or if we are using a lot of existing
            code, we might not want to recompile everything every time we build.
        \item For example: We commonly use the C Standard Library in our programm,
             but we never actually change it.
        \item How can we avoid uneccessary recompilation?
    \end{itemize}
    \end{frame}

\begin{frame}{Static Libraries}
    \begin{itemize}
        \item Remember that compilation involved multiple steps:
        \begin{enumerate}
            \item Preprocessing
            \item Compiling
            \item (Assembling)
            \item Linking
        \end{enumerate}
        \item Steps 1-3 run for every file. The output doesn't change, unless
            the C file (or the included headers) change.
        \item So should we just link all the .o / .obj files of libc and other
            dependencies every time? Doesn't that get annoying?
        \item GNU/Linux allows us to bundle multiple .o files into a bundle 
            (think .zip) called an archive (.a)
        \item On Windows multiple .obj's can be bundled into a .LIB file
        \item We call both .a and .LIB static libraries, since they define
            a bunch of functions (and global variables) that we can use 
            in our applications. 
    \end{itemize}
\end{frame}

\begin{frame}[fragile]{Creating a Static Library}
    \begin{codeblock}
//hello.c
#include <stdio.h>
void hello_can_you_hear_me(void){
    puts("Hello from the other side!");
}
    \end{codeblock}
    \begin{lstlisting}
$ gcc -c hello.c 
$ ar rcs libhello.a hello.o     
    \end{lstlisting}
    \begin{itemize}
        \item \code{-c} makes gcc stop after the compilation step and generate
            object files instead of an executable
        \item  the ar utility (with the parameters rcs) creates our archive
        \item we could have appended multiple files for both programs 
        \item prefixing libraries with \code{"lib"} is a typical linux convention
    \end{itemize}
\end{frame}


\begin{frame}[fragile]{Using a Static Library}
    \begin{codeblock}
//main.c
void hello_can_you_hear_me(void); //forward declaration
int main(){
    hello_can_you_hear_me();
    return 0;
}
    \end{codeblock}
    \begin{lstlisting}
$ gcc main.c test.a 
$ ./a.out
Hello from the other side!
$
    \end{lstlisting}
\end{frame}

\subsection{Dynamic Libraries}
\begin{frame}{Dynamic Libraries}
    \begin{itemize}
        \item In a normal Desktop Operating System,
            many processes run at the same time.
        \item For every Process, the OS loads the code inside the executable
            into main memory.
        \item If one executable is launched multiple times, the physical memory
            for the code can be shared.
        \item But if all Libraries were statically linked into the executables,
            commonly used libraries like \code{libc} would end up being stored 
            in RAM many times, creating a lot of redundancy. 
    \end{itemize}
    \bigskip
    \begin{itemize}
        \item Also, when the Code for a library gets updates (bugfixes, 
            performance improvements, etc.), all executables (and libraries)
            depending on that library would need to be recompiled and redistributed
            to the users.
        \item And by the way, how do you create optional modules / plugins?
    \end{itemize}
\end{frame}

\begin{frame}[fragile]{Dynamic Libraries}
    \begin{itemize}
        \item Dynamic libraries are completely different from static libraries,
            in that they get loaded during runtime of the program.
        \item The Application specifies the name of the dnymic libraries 
            its wants to use, and during runtime these symbols get loaded by searching
            for files with the specied names and searching the requested symbols in them.  
    \end{itemize}
    \bigskip
    \begin{codeblock}
#include <dlfcn.h>
int main(){ 
    void *dl_handle = dlopen ("libm.so", RTLD_LAZY);
    double (*cosine)(double);
    cosine = dlsym(dl_handle, "cos");
    printf ("%f\n", (*cosine)(2.0));
    dlclose(handle);
}
    \end{codeblock}
\end{frame}

\begin{frame}[fragile]{Dynamic Libraries}
    \begin{lstlisting}
$ gcc -shared test.c -o libtest.so 
$ gcc main.c -L . -l test
$ export LD_LIBRARY_PATH=./:$LD_LIBRARY_PATH
$./a.out
Hello from the other side!
$
    \end{lstlisting}
    \begin{itemize}
        \item \code{-shared} requests building a shared library
        \item \code{-L} adds a directory to search for libraries (at compile time)
        \item \code{-l} adds a library name to search for (also later used at runtime)
        \item we must omit the leading \code{"lib"} so our library will be found
        \item \code{LD_LIBRARY_PATH} is an environment variable for adding
            additional directories that are searched for dynamic libraries
            (ld is the dynamic linker)
        \item we must add \code{"."} so the current directory is searched
    \end{itemize}
\end{frame}

\begin{frame}[fragile]{Dynamic Libraries Trivia}
\begin{itemize}
    \item another useful variable is \code{LD_PRELOAD}, which allows us to
    preload arbitrary libraries and therefore symbols for our executable
    (e.g. used by strace, can have security implications)

    \item Windows works a bit different here, you have to specify the functions 
        which you want to export for the dynamic library by prepending 
        \code{__declspec(dllexport)} to every function,
        and forward declare them them with \code{__declspec(dllimport)}
    \item this is usually done using a macro like \code{DLL_EXPORT} in header files
        which changes meaning based on pp directives 
        (e.g. expands to nothing on Unix)
    \item Windows also requires you to link a static wrapper library to your application
        for all DLLs. This wrapper .LIB gets automatically created when the DLL is compiled
    \item On the upside, windows automatically searches for DLLS in the application's directory 
\end{itemize}
\end{frame}
\end{document}
