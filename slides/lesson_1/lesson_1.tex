% document
\documentclass[10pt,graphics,aspectratio=169,table]{beamer}
\usepackage{listings}
\usepackage{csquotes}
\usepackage{hyperref}
% theme
\usetheme{metropolis}
% packages
\title{Lesson 1}
\author{Christian Schwarz, Jakob Krebs}

\begin{document}
\maketitle
\begin{frame}{Roadmap}
  \tableofcontents
\end{frame}
\section{Introduction}
\subsection{}
\begin{frame}{Getting started}
  \begin{itemize}
  \item we will use mostly linux
  \item all slides and examples will be available on github
    \url{https://github.com/jkrbs/c_lessons}
  \item all tasks can be send to us via e-mail and we will prvide feedback
    \url{c-lessons@deutschland.gmbh}
  \item information on our course will be available on \url{TODO}
  \item weekly lesons %% TODO add date
  \end{itemize}
\end{frame}

\begin{frame}{development envirement}
  \begin{itemize}
  \item you can use any editor of your choice
  \item you also can use an ide like vscode, clion, \ldots
  \item we will use a commandline, vim and gcc
  \end{itemize}
  
\end{frame}
\begin{frame}[fragile]{gcc for Unix-based operating systems}
	Ubuntu / Debian:
	\begin{lstlisting}[numbers=none]
    $ sudo apt-get install gcc
  \end{lstlisting}
	\bigskip
	Arch Linux:
	\begin{lstlisting}[numbers=none]
    $ sudo pacman -S gcc
  \end{lstlisting}
	\bigskip
	Mac OS X:
	\begin{lstlisting}[numbers=none]
    $ brew install gcc
  \end{lstlisting}
	\bigskip
	... and you're done ;-)
\end{frame} 
\begin{frame}[fragile]{usage}
  Debug:
  \begin{lstlisting}
    $ gcc -g -Wall -o output input.c
  \end{lstlisting}
  \bigskip
  Release:
  \begin{lstlisting}
    $ gcc -Wall -O2 -o output input1.c input2.c
    $ strip output
  \end{lstlisting}
\end{frame}
\section{Make}
\subsection{}

\begin{frame}{motivation}
  you have seen the gcc commands.
  \bigskip
  there is a better solution
\end{frame}

\begin{frame}[fragile]{Makefile}
  \begin{lstlisting}
    CC := gcc
    CFLAGS := -Wall
    DFlAGS := $(CFLAGS) -g
    RFLAGS := $(CLAFGS) -O2
    EXE := our_binary_name
    SRC := $(shell find src/ -iname '*.c')

    # prevent make from treating targets as file names
    .PHONY release clean debug 

    release: 
    $(CC) $(RFLAGS) -o $(EXE) $(SRC)

    debug:
    $(CC) $(DFLAGS) -o $(EXE) $(SRC)

    clean:
    @rm $(EXE)
    
  \end{lstlisting}
\end{frame}

\begin{frame}[fragile]{usage}
  write your makefile in a file called \enquote{Makefile} in the root directory
  of your project
  
  \begin{lstlisting}
    $ make # in the root directory of your project
  \end{lstlisting}
  
\end{frame}
\section{Task}
\subsection{}
\begin{frame}{Task}
  Write a c program which finds prime numbers to a given maximum.
  And write a makefile to build it.
\end{frame}
\begin{frame}{additional tasks}
  \begin{itemize}
  \item make the program dynamically allocate memory and print primes until
    it is terminated
  \item use a faster prime finding algorithm like erastosthenes sieve
  \item write your output to a file given as a first commandline
    argument and display a progress on stdout
  \end{itemize}
\end{frame}
\begin{frame}[fragile]{help for additional tasks: memory allocation}
  \begin{lstlisting}
    void* malloc(size_t size);
    void free(void* ptr);
  \end{lstlisting}
  example:
  \begin{lstlisting}
    #include <stdio.h>
    int main() {
      int* foo = malloc(100* sizeof(int));
      for(int i = 0; i < 100; i++) {
        foo[i] = i;
      }
      free(foo);
      return 0;
    }
  \end{lstlisting}
\end{frame}
\begin{frame}[fragile]{help for additional tasks: file io}
  \begin{lstlisting}
    FILE* fopen(const char* path, const char* permissions);
    size_t fwrite(const void* ptr, size_t size, size_t cnt, 
    FILE* stream);
    size_t fread(void* ptr, size_t size, size_t cnt, 
    FILE* stream);
    int fclose(FILE* filetoclose)
  \end{lstlisting}
  example:
  \begin{lstlisting}
    int main() {
      FILE* f = fopen("foo.txt", "w");
      char* data = "lololol\n";
      size_t data_size = 8;
      assert(fwrite(data, data_size, 1, f) != data_size);
      fclose(f);
      return 0;
    }
  \end{lstlisting}

\end{frame}
\end{document}
