\documentclass[10pt,graphics,aspectratio=169,table]{beamer}
\usepackage{../code}
\usetheme{metropolis}
\begin{document}
\begin{frame}[fragile]{Task 3: Control Structures; 3.1a For}
    \begin{codeblock}
int i = 0; 
while (i < 10) {
    puts("Hello");
    i++;
}
    \end{codeblock}
\end{frame}

\begin{frame}[fragile]{Task 3: Control Structures; 3.1b for}
    \begin{codeblock}
puts("Hello");
puts("Hello");
puts("Hello");
puts("Hello");
puts("Hello");
puts("Hello");
puts("Hello");
puts("Hello");
puts("Hello");
puts("Hello");
    \end{codeblock}

    or just 
    \begin{codeblock}
puts(
    "Hello\nHello\nHello\nHello\nHello\n"
    "Hello\nHello\nHello\nHello\nHello"
);
    \end{codeblock}
\end{frame}

\begin{frame}[fragile]{Task 3: Control Structures; 3.2 Do While}
    \begin{codeblock}
int x = 0;
while(x < 9){
    i++;
    printf("%i\n", i);
    i++; 
}
    \end{codeblock}

A little trick question. We learned that there is no difference between
while and do while after the first iteration passes.
\end{frame}

\begin{frame}[fragile]{Task 3: Control Structures; 3.3 Switch}
    \begin{codeblock}
int x = i;
while(x < 100){
    int fbstate = i % 3 + 2 * (i % 5);
    if (fbstate == 0) printf("%i\n", i);
    else if (fbstate == 1) puts("Fizz!");
    else if(fbstate == 2) puts("Buzz!");
    else if(fbstate == 3) puts("FizzBuzz!"); 
    i++; 
}
    \end{codeblock}

    We could just use else on the last branch, 
    but the task doesn't use default either. (Good code would, though).
\end{frame}
\end{document}
