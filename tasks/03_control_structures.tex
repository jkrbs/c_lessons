\documentclass[10pt,graphics,aspectratio=169,table]{beamer}
\usepackage{../code}
\usetheme{metropolis}
\begin{document}
\begin{frame}[fragile]{Task 3: Control Structures}
This week we learned about \code{for}, \code{do..while}, and \code{switch}.
But all of these concepts are just syntactic sugar for the basics that 
we learned about last time: \code{if} and \code{while}.

Try to reproduce the output of the following three blocks of code exactly 
using only the control structures \code{if}, \code{else} and \code{while}.

You may introduce as many variables and statements as yout want, 
but try to aim for the shortest possible solution. 
\end{frame}

\begin{frame}[fragile]{Task 3.1: For}
\begin{codeblock}
for(int i = 0; i< 10; i++){
    puts("Hello");
}
\end{codeblock}

It would be kind of dumb, but could we do it without
any control structures at all?

\end{frame}
\begin{frame}[fragile]{Task 3.2: Do While}
\begin{codeblock}
    int x = 0;
    do{
        i++;
        printf("%i\n", i);
        i++;
    }while(x < 9);
\end{codeblock}

\end{frame}
\begin{frame}[fragile]{Task 3.3: Switch}
\begin{codeblock}
for(int i = 0;i<100;i++){
    switch(i % 3 + 2 * (i % 5)){
        case 0: printf("%i\n"); break;
        case 1: puts("Fizz!"); break;
        case 2: puts("Buzz!"); break;
        case 3: puts("FizzBuzz!"); break;
    }
}
\end{codeblock}

This might look familiar, eh? Try to preserve the logic shown here 
as closely as possible during your conversion. 
Remember: You may intdroduce additional variables.
\end{frame}
\end{document}
