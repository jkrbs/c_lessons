\documentclass[10pt,graphics,aspectratio=169,table]{beamer}
\usepackage{../code}
\usetheme{metropolis}
\begin{document}
\begin{frame}[fragile]{Task 10: Binary Trees}

    \begin{itemize}
        \item Create a Binary Tree Library file \code{tree.c} with a corresponding
            \code{tree.h} that allows one to call \code{tree_insert} to insert 
            an integer into the tree.
            The tree should always remain sorted so that the left child is smaller 
            or equal to the root node, which is smaller than the right node.
            Example:
            \begin{codeblock}
    1  
   / \
  2   5
 / \ / \
3  4 6  7
            \end{codeblock}

      
        \item If you like, you can start out with the code that we created 
              two weeks ago during the lesson:
        \tiny
        \url{https://jkrbs.github.io/c_lessons/task_solution_sources/intermediate_4_binary_tree.c}
        
        \normalsize

        \item Add a \code{tree_print} method that prints the tree as an ordered list.

        \item Create a small main file that uses \code{tree.h},
              Adds 1 to 10 to a the tree and then prints the tree.
              
        \item Bonus: Can you write a method that prints the tree as shown above?
    \end{itemize}

\end{frame}

\begin{frame}[fragile]{Task 9: Hints}
    \begin{itemize}
        \item The new \code{struct list_node} could look like this:
            \begin{codeblock}
typedef struct list_node {
    int value;    
    struct list_node* next;
    struct list_node* previous;
} list_node;
            \end{codeblock}

        \item \code{previous} of the first node is \code{NULL}
        \item \code{next} of the last node is \code{NULL} 
        \item If you forgot how the pointers have to point, 
              look at the slides again, maybe they can help.
        
        \item Bonus Hint: Recursion :).
\end{itemize}
   
    
\end{frame}
\end{document}
